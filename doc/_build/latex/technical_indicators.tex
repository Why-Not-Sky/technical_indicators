% Generated by Sphinx.
\def\sphinxdocclass{report}
\documentclass[letterpaper,10pt,english]{sphinxmanual}
\usepackage[utf8]{inputenc}
\DeclareUnicodeCharacter{00A0}{\nobreakspace}
\usepackage{cmap}
\usepackage[T1]{fontenc}
\usepackage{babel}
\usepackage{times}
\usepackage[Bjarne]{fncychap}
\usepackage{longtable}
\usepackage{sphinx}
\usepackage{multirow}


\title{technical\_indicators Documentation}
\date{May 30, 2014}
\release{0.0.7}
\author{Joao Matos}
\newcommand{\sphinxlogo}{}
\renewcommand{\releasename}{Release}
\makeindex

\makeatletter
\def\PYG@reset{\let\PYG@it=\relax \let\PYG@bf=\relax%
    \let\PYG@ul=\relax \let\PYG@tc=\relax%
    \let\PYG@bc=\relax \let\PYG@ff=\relax}
\def\PYG@tok#1{\csname PYG@tok@#1\endcsname}
\def\PYG@toks#1+{\ifx\relax#1\empty\else%
    \PYG@tok{#1}\expandafter\PYG@toks\fi}
\def\PYG@do#1{\PYG@bc{\PYG@tc{\PYG@ul{%
    \PYG@it{\PYG@bf{\PYG@ff{#1}}}}}}}
\def\PYG#1#2{\PYG@reset\PYG@toks#1+\relax+\PYG@do{#2}}

\expandafter\def\csname PYG@tok@gd\endcsname{\def\PYG@tc##1{\textcolor[rgb]{0.63,0.00,0.00}{##1}}}
\expandafter\def\csname PYG@tok@gu\endcsname{\let\PYG@bf=\textbf\def\PYG@tc##1{\textcolor[rgb]{0.50,0.00,0.50}{##1}}}
\expandafter\def\csname PYG@tok@gt\endcsname{\def\PYG@tc##1{\textcolor[rgb]{0.00,0.27,0.87}{##1}}}
\expandafter\def\csname PYG@tok@gs\endcsname{\let\PYG@bf=\textbf}
\expandafter\def\csname PYG@tok@gr\endcsname{\def\PYG@tc##1{\textcolor[rgb]{1.00,0.00,0.00}{##1}}}
\expandafter\def\csname PYG@tok@cm\endcsname{\let\PYG@it=\textit\def\PYG@tc##1{\textcolor[rgb]{0.25,0.50,0.56}{##1}}}
\expandafter\def\csname PYG@tok@vg\endcsname{\def\PYG@tc##1{\textcolor[rgb]{0.73,0.38,0.84}{##1}}}
\expandafter\def\csname PYG@tok@m\endcsname{\def\PYG@tc##1{\textcolor[rgb]{0.13,0.50,0.31}{##1}}}
\expandafter\def\csname PYG@tok@mh\endcsname{\def\PYG@tc##1{\textcolor[rgb]{0.13,0.50,0.31}{##1}}}
\expandafter\def\csname PYG@tok@cs\endcsname{\def\PYG@tc##1{\textcolor[rgb]{0.25,0.50,0.56}{##1}}\def\PYG@bc##1{\setlength{\fboxsep}{0pt}\colorbox[rgb]{1.00,0.94,0.94}{\strut ##1}}}
\expandafter\def\csname PYG@tok@ge\endcsname{\let\PYG@it=\textit}
\expandafter\def\csname PYG@tok@vc\endcsname{\def\PYG@tc##1{\textcolor[rgb]{0.73,0.38,0.84}{##1}}}
\expandafter\def\csname PYG@tok@il\endcsname{\def\PYG@tc##1{\textcolor[rgb]{0.13,0.50,0.31}{##1}}}
\expandafter\def\csname PYG@tok@go\endcsname{\def\PYG@tc##1{\textcolor[rgb]{0.20,0.20,0.20}{##1}}}
\expandafter\def\csname PYG@tok@cp\endcsname{\def\PYG@tc##1{\textcolor[rgb]{0.00,0.44,0.13}{##1}}}
\expandafter\def\csname PYG@tok@gi\endcsname{\def\PYG@tc##1{\textcolor[rgb]{0.00,0.63,0.00}{##1}}}
\expandafter\def\csname PYG@tok@gh\endcsname{\let\PYG@bf=\textbf\def\PYG@tc##1{\textcolor[rgb]{0.00,0.00,0.50}{##1}}}
\expandafter\def\csname PYG@tok@ni\endcsname{\let\PYG@bf=\textbf\def\PYG@tc##1{\textcolor[rgb]{0.84,0.33,0.22}{##1}}}
\expandafter\def\csname PYG@tok@nl\endcsname{\let\PYG@bf=\textbf\def\PYG@tc##1{\textcolor[rgb]{0.00,0.13,0.44}{##1}}}
\expandafter\def\csname PYG@tok@nn\endcsname{\let\PYG@bf=\textbf\def\PYG@tc##1{\textcolor[rgb]{0.05,0.52,0.71}{##1}}}
\expandafter\def\csname PYG@tok@no\endcsname{\def\PYG@tc##1{\textcolor[rgb]{0.38,0.68,0.84}{##1}}}
\expandafter\def\csname PYG@tok@na\endcsname{\def\PYG@tc##1{\textcolor[rgb]{0.25,0.44,0.63}{##1}}}
\expandafter\def\csname PYG@tok@nb\endcsname{\def\PYG@tc##1{\textcolor[rgb]{0.00,0.44,0.13}{##1}}}
\expandafter\def\csname PYG@tok@nc\endcsname{\let\PYG@bf=\textbf\def\PYG@tc##1{\textcolor[rgb]{0.05,0.52,0.71}{##1}}}
\expandafter\def\csname PYG@tok@nd\endcsname{\let\PYG@bf=\textbf\def\PYG@tc##1{\textcolor[rgb]{0.33,0.33,0.33}{##1}}}
\expandafter\def\csname PYG@tok@ne\endcsname{\def\PYG@tc##1{\textcolor[rgb]{0.00,0.44,0.13}{##1}}}
\expandafter\def\csname PYG@tok@nf\endcsname{\def\PYG@tc##1{\textcolor[rgb]{0.02,0.16,0.49}{##1}}}
\expandafter\def\csname PYG@tok@si\endcsname{\let\PYG@it=\textit\def\PYG@tc##1{\textcolor[rgb]{0.44,0.63,0.82}{##1}}}
\expandafter\def\csname PYG@tok@s2\endcsname{\def\PYG@tc##1{\textcolor[rgb]{0.25,0.44,0.63}{##1}}}
\expandafter\def\csname PYG@tok@vi\endcsname{\def\PYG@tc##1{\textcolor[rgb]{0.73,0.38,0.84}{##1}}}
\expandafter\def\csname PYG@tok@nt\endcsname{\let\PYG@bf=\textbf\def\PYG@tc##1{\textcolor[rgb]{0.02,0.16,0.45}{##1}}}
\expandafter\def\csname PYG@tok@nv\endcsname{\def\PYG@tc##1{\textcolor[rgb]{0.73,0.38,0.84}{##1}}}
\expandafter\def\csname PYG@tok@s1\endcsname{\def\PYG@tc##1{\textcolor[rgb]{0.25,0.44,0.63}{##1}}}
\expandafter\def\csname PYG@tok@gp\endcsname{\let\PYG@bf=\textbf\def\PYG@tc##1{\textcolor[rgb]{0.78,0.36,0.04}{##1}}}
\expandafter\def\csname PYG@tok@sh\endcsname{\def\PYG@tc##1{\textcolor[rgb]{0.25,0.44,0.63}{##1}}}
\expandafter\def\csname PYG@tok@ow\endcsname{\let\PYG@bf=\textbf\def\PYG@tc##1{\textcolor[rgb]{0.00,0.44,0.13}{##1}}}
\expandafter\def\csname PYG@tok@sx\endcsname{\def\PYG@tc##1{\textcolor[rgb]{0.78,0.36,0.04}{##1}}}
\expandafter\def\csname PYG@tok@bp\endcsname{\def\PYG@tc##1{\textcolor[rgb]{0.00,0.44,0.13}{##1}}}
\expandafter\def\csname PYG@tok@c1\endcsname{\let\PYG@it=\textit\def\PYG@tc##1{\textcolor[rgb]{0.25,0.50,0.56}{##1}}}
\expandafter\def\csname PYG@tok@kc\endcsname{\let\PYG@bf=\textbf\def\PYG@tc##1{\textcolor[rgb]{0.00,0.44,0.13}{##1}}}
\expandafter\def\csname PYG@tok@c\endcsname{\let\PYG@it=\textit\def\PYG@tc##1{\textcolor[rgb]{0.25,0.50,0.56}{##1}}}
\expandafter\def\csname PYG@tok@mf\endcsname{\def\PYG@tc##1{\textcolor[rgb]{0.13,0.50,0.31}{##1}}}
\expandafter\def\csname PYG@tok@err\endcsname{\def\PYG@bc##1{\setlength{\fboxsep}{0pt}\fcolorbox[rgb]{1.00,0.00,0.00}{1,1,1}{\strut ##1}}}
\expandafter\def\csname PYG@tok@kd\endcsname{\let\PYG@bf=\textbf\def\PYG@tc##1{\textcolor[rgb]{0.00,0.44,0.13}{##1}}}
\expandafter\def\csname PYG@tok@ss\endcsname{\def\PYG@tc##1{\textcolor[rgb]{0.32,0.47,0.09}{##1}}}
\expandafter\def\csname PYG@tok@sr\endcsname{\def\PYG@tc##1{\textcolor[rgb]{0.14,0.33,0.53}{##1}}}
\expandafter\def\csname PYG@tok@mo\endcsname{\def\PYG@tc##1{\textcolor[rgb]{0.13,0.50,0.31}{##1}}}
\expandafter\def\csname PYG@tok@mi\endcsname{\def\PYG@tc##1{\textcolor[rgb]{0.13,0.50,0.31}{##1}}}
\expandafter\def\csname PYG@tok@kn\endcsname{\let\PYG@bf=\textbf\def\PYG@tc##1{\textcolor[rgb]{0.00,0.44,0.13}{##1}}}
\expandafter\def\csname PYG@tok@o\endcsname{\def\PYG@tc##1{\textcolor[rgb]{0.40,0.40,0.40}{##1}}}
\expandafter\def\csname PYG@tok@kr\endcsname{\let\PYG@bf=\textbf\def\PYG@tc##1{\textcolor[rgb]{0.00,0.44,0.13}{##1}}}
\expandafter\def\csname PYG@tok@s\endcsname{\def\PYG@tc##1{\textcolor[rgb]{0.25,0.44,0.63}{##1}}}
\expandafter\def\csname PYG@tok@kp\endcsname{\def\PYG@tc##1{\textcolor[rgb]{0.00,0.44,0.13}{##1}}}
\expandafter\def\csname PYG@tok@w\endcsname{\def\PYG@tc##1{\textcolor[rgb]{0.73,0.73,0.73}{##1}}}
\expandafter\def\csname PYG@tok@kt\endcsname{\def\PYG@tc##1{\textcolor[rgb]{0.56,0.13,0.00}{##1}}}
\expandafter\def\csname PYG@tok@sc\endcsname{\def\PYG@tc##1{\textcolor[rgb]{0.25,0.44,0.63}{##1}}}
\expandafter\def\csname PYG@tok@sb\endcsname{\def\PYG@tc##1{\textcolor[rgb]{0.25,0.44,0.63}{##1}}}
\expandafter\def\csname PYG@tok@k\endcsname{\let\PYG@bf=\textbf\def\PYG@tc##1{\textcolor[rgb]{0.00,0.44,0.13}{##1}}}
\expandafter\def\csname PYG@tok@se\endcsname{\let\PYG@bf=\textbf\def\PYG@tc##1{\textcolor[rgb]{0.25,0.44,0.63}{##1}}}
\expandafter\def\csname PYG@tok@sd\endcsname{\let\PYG@it=\textit\def\PYG@tc##1{\textcolor[rgb]{0.25,0.44,0.63}{##1}}}

\def\PYGZbs{\char`\\}
\def\PYGZus{\char`\_}
\def\PYGZob{\char`\{}
\def\PYGZcb{\char`\}}
\def\PYGZca{\char`\^}
\def\PYGZam{\char`\&}
\def\PYGZlt{\char`\<}
\def\PYGZgt{\char`\>}
\def\PYGZsh{\char`\#}
\def\PYGZpc{\char`\%}
\def\PYGZdl{\char`\$}
\def\PYGZhy{\char`\-}
\def\PYGZsq{\char`\'}
\def\PYGZdq{\char`\"}
\def\PYGZti{\char`\~}
% for compatibility with earlier versions
\def\PYGZat{@}
\def\PYGZlb{[}
\def\PYGZrb{]}
\makeatother

\begin{document}

\maketitle
\tableofcontents
\phantomsection\label{index::doc}


This module provides some technical indicators for analysing stocks.

When I can I will add more.

If anyone wishes to contribute with new code or corrections/suggestions, feel free.

\textbf{Features:}
\begin{quote}

Relative Strength Index (RSI), ROC, MA envelopes

Simple Moving Average (SMA), Weighted Moving Average (WMA), Exponential Moving Average (EMA)

Bollinger Bands (BB), Bollinger Bandwidth, \%B
\end{quote}

\textbf{Dependencies:}

It requires numpy.

This module was done and tested under Windows with Python 2.7.3 and numpy 1.6.1.

\textbf{Contents:}


\chapter{Reference}
\label{reference:welcome-to-technical-indicators-s-documentation}\label{reference::doc}\label{reference:reference}

\section{technical\_indicators}
\label{reference:technical-indicators}\label{reference:module-technical_indicators.technical_indicators}\index{technical\_indicators.technical\_indicators (module)}
This module provides some technical indicators for analysing stocks.

When I can I will add more.
If anyone wishes to contribute with new code or corrections/suggestions, feel
free.

Features:
\begin{quote}

Relative Strength Index (RSI), ROC, MA envelopes
Simple Moving Average (SMA), Weighted Moving Average (WMA), Exponential
Moving Average (EMA)
Bollinger Bands (BB), Bollinger Bandwidth, \%B
\end{quote}

Dependencies:

It requires numpy.
This module was tested under Windows with Python 2.7.3 and numpy 1.6.1.
\index{bb() (in module technical\_indicators.technical\_indicators)}

\begin{fulllineitems}
\phantomsection\label{reference:technical_indicators.technical_indicators.bb}\pysiglinewithargsret{\code{technical\_indicators.technical\_indicators.}\bfcode{bb}}{\emph{prices}, \emph{period}, \emph{num\_std\_dev=2.0}}{}
Bollinger bands (BB) are volatility bands placed above and below a moving
average.
Volatility is based on the standard deviation, which changes as volatility
increases and decreases.
The bands automatically widen when volatility increases and narrow when
volatility decreases.
This dynamic nature of Bollinger Bands also means they can be used on
different securities with the standard settings.
For signals, Bollinger Bands can be used to identify M-Tops and W-Bottoms
or to determine the strength of the trend.
Signals derived from narrowing BandWidth are also important.

Bollinger BandWidth is an indicator that derives from Bollinger Bands, and
measures the percentage difference between the upper band and the lower
band.
BandWidth decreases as Bollinger Bands narrow and increases as Bollinger
Bands widen.
Because Bollinger Bands are based on the standard deviation, falling
BandWidth reflects decreasing volatility and rising BandWidth reflects
increasing volatility.

\%B quantifies a security's price relative to the upper and lower Bollinger
Band. There are six basic relationship levels:
\%B equals 1 when price is at the upper band
\%B equals 0 when price is at the lower band
\%B is above 1 when price is above the upper band
\%B is below 0 when price is below the lower band
\%B is above .50 when price is above the middle band (20-day SMA)
\%B is below .50 when price is below the middle band (20-day SMA)

They were developed by John Bollinger.
Bollinger suggests increasing the standard deviation multiplier to 2.1 for
a 50-period SMA and decreasing the standard deviation multiplier to 1.9 for
a 10-period SMA.

\href{http://www.csidata.com/?page\_id=797}{http://www.csidata.com/?page\_id=797}
\href{http://stockcharts.com/school/doku.php?id=chart\_school:technical\_indicators:bollinger\_bands}{http://stockcharts.com/school/doku.php?id=chart\_school:technical\_indicators:bollinger\_bands}
\href{http://stockcharts.com/school/doku.php?id=chart\_school:technical\_indicators:bollinger\_band\_width}{http://stockcharts.com/school/doku.php?id=chart\_school:technical\_indicators:bollinger\_band\_width}
\href{http://stockcharts.com/school/doku.php?id=chart\_school:technical\_indicators:bollinger\_band\_perce}{http://stockcharts.com/school/doku.php?id=chart\_school:technical\_indicators:bollinger\_band\_perce}
\begin{description}
\item[{Input:}] \leavevmode
prices ndarray
period int \textgreater{} 1 and \textless{} len(prices)
num\_std\_dev float \textgreater{} 0.0 (optional and defaults to 2.0)

\item[{Output:}] \leavevmode
bbs ndarray with upper, middle, lower bands, bandwidth, range and \%B

\end{description}

Test:

\begin{Verbatim}[commandchars=\\\{\}]
\PYG{g+gp}{\PYGZgt{}\PYGZgt{}\PYGZgt{} }\PYG{k+kn}{import} \PYG{n+nn}{numpy} \PYG{k+kn}{as} \PYG{n+nn}{np}
\PYG{g+gp}{\PYGZgt{}\PYGZgt{}\PYGZgt{} }\PYG{k+kn}{import} \PYG{n+nn}{technical\PYGZus{}indicators} \PYG{k+kn}{as} \PYG{n+nn}{tai}
\PYG{g+gp}{\PYGZgt{}\PYGZgt{}\PYGZgt{} }\PYG{n}{prices} \PYG{o}{=} \PYG{n}{np}\PYG{o}{.}\PYG{n}{array}\PYG{p}{(}\PYG{p}{[}\PYG{l+m+mf}{86.16}\PYG{p}{,} \PYG{l+m+mf}{89.09}\PYG{p}{,} \PYG{l+m+mf}{88.78}\PYG{p}{,} \PYG{l+m+mf}{90.32}\PYG{p}{,} \PYG{l+m+mf}{89.07}\PYG{p}{,} \PYG{l+m+mf}{91.15}\PYG{p}{,} \PYG{l+m+mf}{89.44}\PYG{p}{,}
\PYG{g+gp}{... }\PYG{l+m+mf}{89.18}\PYG{p}{,} \PYG{l+m+mf}{86.93}\PYG{p}{,} \PYG{l+m+mf}{87.68}\PYG{p}{,} \PYG{l+m+mf}{86.96}\PYG{p}{,} \PYG{l+m+mf}{89.43}\PYG{p}{,} \PYG{l+m+mf}{89.32}\PYG{p}{,} \PYG{l+m+mf}{88.72}\PYG{p}{,} \PYG{l+m+mf}{87.45}\PYG{p}{,} \PYG{l+m+mf}{87.26}\PYG{p}{,} \PYG{l+m+mf}{89.50}\PYG{p}{,}
\PYG{g+gp}{... }\PYG{l+m+mf}{87.90}\PYG{p}{,} \PYG{l+m+mf}{89.13}\PYG{p}{,} \PYG{l+m+mf}{90.70}\PYG{p}{,} \PYG{l+m+mf}{92.90}\PYG{p}{,} \PYG{l+m+mf}{92.98}\PYG{p}{,} \PYG{l+m+mf}{91.80}\PYG{p}{,} \PYG{l+m+mf}{92.66}\PYG{p}{,} \PYG{l+m+mf}{92.68}\PYG{p}{,} \PYG{l+m+mf}{92.30}\PYG{p}{,} \PYG{l+m+mf}{92.77}\PYG{p}{,}
\PYG{g+gp}{... }\PYG{l+m+mf}{92.54}\PYG{p}{,} \PYG{l+m+mf}{92.95}\PYG{p}{,} \PYG{l+m+mf}{93.20}\PYG{p}{,} \PYG{l+m+mf}{91.07}\PYG{p}{,} \PYG{l+m+mf}{89.83}\PYG{p}{,} \PYG{l+m+mf}{89.74}\PYG{p}{,} \PYG{l+m+mf}{90.40}\PYG{p}{,} \PYG{l+m+mf}{90.74}\PYG{p}{,} \PYG{l+m+mf}{88.02}\PYG{p}{,} \PYG{l+m+mf}{88.09}\PYG{p}{,}
\PYG{g+gp}{... }\PYG{l+m+mf}{88.84}\PYG{p}{,} \PYG{l+m+mf}{90.78}\PYG{p}{,} \PYG{l+m+mf}{90.54}\PYG{p}{,} \PYG{l+m+mf}{91.39}\PYG{p}{,} \PYG{l+m+mf}{90.65}\PYG{p}{]}\PYG{p}{)}
\PYG{g+gp}{\PYGZgt{}\PYGZgt{}\PYGZgt{} }\PYG{n}{period} \PYG{o}{=} \PYG{l+m+mi}{20}
\PYG{g+gp}{\PYGZgt{}\PYGZgt{}\PYGZgt{} }\PYG{k}{print}\PYG{p}{(}\PYG{n}{tai}\PYG{o}{.}\PYG{n}{bb}\PYG{p}{(}\PYG{n}{prices}\PYG{p}{,} \PYG{n}{period}\PYG{p}{)}\PYG{p}{)}
\PYG{g+go}{[[  9.12919107e+01   8.87085000e+01   8.61250893e+01   5.82449423e\PYGZhy{}02}
\PYG{g+go}{    5.16682146e+00   6.75671306e\PYGZhy{}03]}
\PYG{g+go}{ [  9.19497209e+01   8.90455000e+01   8.61412791e+01   6.52300429e\PYGZhy{}02}
\PYG{g+go}{    5.80844179e+00   5.07661263e\PYGZhy{}01]}
\PYG{g+go}{ [  9.26132536e+01   8.92400000e+01   8.58667464e+01   7.55995881e\PYGZhy{}02}
\PYG{g+go}{    6.74650724e+00   4.31816571e\PYGZhy{}01]}
\PYG{g+go}{ [  9.29344497e+01   8.93910000e+01   8.58475503e+01   7.92797873e\PYGZhy{}02}
\PYG{g+go}{    7.08689946e+00   6.31086945e\PYGZhy{}01]}
\PYG{g+go}{ [  9.33114122e+01   8.95080000e+01   8.57045878e+01   8.49848539e\PYGZhy{}02}
\PYG{g+go}{    7.60682430e+00   4.42420124e\PYGZhy{}01]}
\PYG{g+go}{ [  9.37270110e+01   8.96885000e+01   8.56499890e+01   9.00563838e\PYGZhy{}02}
\PYG{g+go}{    8.07702198e+00   6.80945403e\PYGZhy{}01]}
\PYG{g+go}{ [  9.38972812e+01   8.97460000e+01   8.55947188e+01   9.25117832e\PYGZhy{}02}
\PYG{g+go}{    8.30256250e+00   4.63143909e\PYGZhy{}01]}
\PYG{g+go}{ [  9.42636418e+01   8.99125000e+01   8.55613582e+01   9.67861377e\PYGZhy{}02}
\PYG{g+go}{    8.70228361e+00   4.15826692e\PYGZhy{}01]}
\PYG{g+go}{ [  9.45630193e+01   9.00805000e+01   8.55979807e+01   9.95225220e\PYGZhy{}02}
\PYG{g+go}{    8.96503854e+00   1.48579313e\PYGZhy{}01]}
\PYG{g+go}{ [  9.47851634e+01   9.03815000e+01   8.59778366e+01   9.74461225e\PYGZhy{}02}
\PYG{g+go}{    8.80732672e+00   1.93266744e\PYGZhy{}01]}
\PYG{g+go}{ [  9.50411874e+01   9.06575000e+01   8.62738126e+01   9.67087637e\PYGZhy{}02}
\PYG{g+go}{    8.76737475e+00   7.82660026e\PYGZhy{}02]}
\PYG{g+go}{ [  9.49062071e+01   9.08630000e+01   8.68197929e+01   8.89956780e\PYGZhy{}02}
\PYG{g+go}{    8.08641429e+00   3.22789193e\PYGZhy{}01]}
\PYG{g+go}{ [  9.49015375e+01   9.08830000e+01   8.68644625e+01   8.84332063e\PYGZhy{}02}
\PYG{g+go}{    8.03707509e+00   3.05526266e\PYGZhy{}01]}
\PYG{g+go}{ [  9.48939343e+01   9.09040000e+01   8.69140657e+01   8.77834713e\PYGZhy{}02}
\PYG{g+go}{    7.97986867e+00   2.26311285e\PYGZhy{}01]}
\PYG{g+go}{ [  9.48594576e+01   9.09880000e+01   8.71165424e+01   8.50982021e\PYGZhy{}02}
\PYG{g+go}{    7.74291521e+00   4.30661576e\PYGZhy{}02]}
\PYG{g+go}{ [  9.46722663e+01   9.11525000e+01   8.76327337e+01   7.72280810e\PYGZhy{}02}
\PYG{g+go}{    7.03953265e+00  \PYGZhy{}5.29486389e\PYGZhy{}02]}
\PYG{g+go}{ [  9.45543042e+01   9.11905000e+01   8.78266958e+01   7.37753219e\PYGZhy{}02}
\PYG{g+go}{    6.72760849e+00   2.48722001e\PYGZhy{}01]}
\PYG{g+go}{ [  9.46761721e+01   9.11200000e+01   8.75638279e+01   7.80546993e\PYGZhy{}02}
\PYG{g+go}{    7.11234420e+00   4.72660054e\PYGZhy{}02]}
\PYG{g+go}{ [  9.45733946e+01   9.11670000e+01   8.77606054e+01   7.47286754e\PYGZhy{}02}
\PYG{g+go}{    6.81278915e+00   2.01003516e\PYGZhy{}01]}
\PYG{g+go}{ [  9.45322396e+01   9.12495000e+01   8.79667604e+01   7.19508503e\PYGZhy{}02}
\PYG{g+go}{    6.56547911e+00   4.16304661e\PYGZhy{}01]}
\PYG{g+go}{ [  9.45303313e+01   9.12415000e+01   8.79526687e+01   7.20906879e\PYGZhy{}02}
\PYG{g+go}{    6.57766250e+00   7.52141243e\PYGZhy{}01]}
\PYG{g+go}{ [  9.43672335e+01   9.11660000e+01   8.79647665e+01   7.02286710e\PYGZhy{}02}
\PYG{g+go}{    6.40246702e+00   7.83328285e\PYGZhy{}01]}
\PYG{g+go}{ [  9.41460689e+01   9.10495000e+01   8.79529311e+01   6.80194599e\PYGZhy{}02}
\PYG{g+go}{    6.19313782e+00   6.21182512e\PYGZhy{}01]]}
\end{Verbatim}

\end{fulllineitems}

\index{ema() (in module technical\_indicators.technical\_indicators)}

\begin{fulllineitems}
\phantomsection\label{reference:technical_indicators.technical_indicators.ema}\pysiglinewithargsret{\code{technical\_indicators.technical\_indicators.}\bfcode{ema}}{\emph{prices}, \emph{period}, \emph{ema\_type=0}}{}
Exponencial Moving Average (EMA) are used to smooth the data in an array to
help eliminate noise and identify trends.
Exponential moving averages reduce the lag by applying more weight to
recent prices.
The weighting applied to the most recent price depends on the number of
periods in the moving average.

They do not predict price direction, but can be used to identify the
direction of the trend or define potential support and resistance levels.

EMA type 0
EMAn = w.Pn + (1 - w).EMAn-1
EMAn = EMAn-1 + w.(Pn - EMAn-1)
EMAn = w.Pn + w.(1 - w).Pn-1 + w.(1 - w)\textasciicircum{}2.Pn-2 + ... +
w.(1 - w)\textasciicircum{}(n-1).P1 + w.(1 - w)\textasciicircum{}n.EMA0
where w = 2 / (n + 1) and EMA0 = mean(oldest period)
or
EMAn = w.EMAn-1 + (1 - w).Pn
where w = 1 - 2 / (n + 1) and Pn is the most recent price
and EMA0 = mean(oldest period)

EMA type 1
The above formulas with EMA0 = P1 (oldest price)

EMA type 2
EMA = (Pn + w.Pn-1 + w\textasciicircum{}2.Pn-2 + w\textasciicircum{}3.Pn-3 + ... ) / K
where K = 1 + w + w\textasciicircum{}2 + ... = 1 / (1 - w) and Pn is the most recent price
and w = 2 / (N + 1)

\href{http://www.financialwebring.org/gummy-stuff/MA-stuff.htm}{http://www.financialwebring.org/gummy-stuff/MA-stuff.htm}

\href{http://www.csidata.com/?page\_id=797}{http://www.csidata.com/?page\_id=797}
\href{http://stockcharts.com/school/doku.php?st=moving+average\&id=chart\_school:technical\_indicators:moving\_averages}{http://stockcharts.com/school/doku.php?st=moving+average\&id=chart\_school:technical\_indicators:moving\_averages}
\begin{description}
\item[{Input:}] \leavevmode
prices ndarray
period int \textgreater{} 1 and \textless{} len(prices)
ema\_type can be 0, 1 or 2

\item[{Output:}] \leavevmode
emas ndarray

\end{description}

Tests:

\begin{Verbatim}[commandchars=\\\{\}]
\PYG{g+gp}{\PYGZgt{}\PYGZgt{}\PYGZgt{} }\PYG{k+kn}{import} \PYG{n+nn}{numpy} \PYG{k+kn}{as} \PYG{n+nn}{np}
\PYG{g+gp}{\PYGZgt{}\PYGZgt{}\PYGZgt{} }\PYG{k+kn}{import} \PYG{n+nn}{technical\PYGZus{}indicators} \PYG{k+kn}{as} \PYG{n+nn}{tai}
\PYG{g+gp}{\PYGZgt{}\PYGZgt{}\PYGZgt{} }\PYG{n}{prices} \PYG{o}{=} \PYG{n}{np}\PYG{o}{.}\PYG{n}{array}\PYG{p}{(}\PYG{p}{[}\PYG{l+m+mf}{22.27}\PYG{p}{,} \PYG{l+m+mf}{22.19}\PYG{p}{,} \PYG{l+m+mf}{22.08}\PYG{p}{,} \PYG{l+m+mf}{22.17}\PYG{p}{,} \PYG{l+m+mf}{22.18}\PYG{p}{,} \PYG{l+m+mf}{22.13}\PYG{p}{,} \PYG{l+m+mf}{22.23}\PYG{p}{,}
\PYG{g+gp}{... }\PYG{l+m+mf}{22.43}\PYG{p}{,} \PYG{l+m+mf}{22.24}\PYG{p}{,} \PYG{l+m+mf}{22.29}\PYG{p}{,} \PYG{l+m+mf}{22.15}\PYG{p}{,} \PYG{l+m+mf}{22.39}\PYG{p}{,} \PYG{l+m+mf}{22.38}\PYG{p}{,} \PYG{l+m+mf}{22.61}\PYG{p}{,} \PYG{l+m+mf}{23.36}\PYG{p}{,} \PYG{l+m+mf}{24.05}\PYG{p}{,} \PYG{l+m+mf}{23.75}\PYG{p}{,}
\PYG{g+gp}{... }\PYG{l+m+mf}{23.83}\PYG{p}{,} \PYG{l+m+mf}{23.95}\PYG{p}{,} \PYG{l+m+mf}{23.63}\PYG{p}{,} \PYG{l+m+mf}{23.82}\PYG{p}{,} \PYG{l+m+mf}{23.87}\PYG{p}{,} \PYG{l+m+mf}{23.65}\PYG{p}{,} \PYG{l+m+mf}{23.19}\PYG{p}{,} \PYG{l+m+mf}{23.10}\PYG{p}{,} \PYG{l+m+mf}{23.33}\PYG{p}{,} \PYG{l+m+mf}{22.68}\PYG{p}{,}
\PYG{g+gp}{... }\PYG{l+m+mf}{23.10}\PYG{p}{,} \PYG{l+m+mf}{22.40}\PYG{p}{,} \PYG{l+m+mf}{22.17}\PYG{p}{]}\PYG{p}{)}
\PYG{g+gp}{\PYGZgt{}\PYGZgt{}\PYGZgt{} }\PYG{n}{period} \PYG{o}{=} \PYG{l+m+mi}{10}
\PYG{g+gp}{\PYGZgt{}\PYGZgt{}\PYGZgt{} }\PYG{k}{print}\PYG{p}{(}\PYG{n}{tai}\PYG{o}{.}\PYG{n}{ema}\PYG{p}{(}\PYG{n}{prices}\PYG{p}{,} \PYG{n}{period}\PYG{p}{)}\PYG{p}{)}
\PYG{g+go}{[ 22.221       22.20809091  22.24116529  22.26640796  22.32887924}
\PYG{g+go}{  22.51635574  22.79520015  22.96880013  23.12538192  23.27531248}
\PYG{g+go}{  23.33980112  23.42711001  23.50763546  23.53351992  23.47106176}
\PYG{g+go}{  23.40359598  23.39021489  23.26108491  23.23179675  23.08056097}
\PYG{g+go}{  22.91500443]}
\PYG{g+gp}{\PYGZgt{}\PYGZgt{}\PYGZgt{} }\PYG{k}{print}\PYG{p}{(}\PYG{n}{tai}\PYG{o}{.}\PYG{n}{ema}\PYG{p}{(}\PYG{n}{prices}\PYG{p}{,} \PYG{n}{period}\PYG{p}{,} \PYG{l+m+mi}{1}\PYG{p}{)}\PYG{p}{)}
\PYG{g+go}{[ 22.27        22.25545455  22.22355372  22.21381668  22.20766819}
\PYG{g+go}{  22.1935467   22.20017457  22.24196102  22.24160447  22.25040366}
\PYG{g+go}{  22.23214845  22.26084873  22.2825126   22.34205576  22.52713653}
\PYG{g+go}{  22.8040208   22.97601702  23.13128665  23.28014362  23.34375387}
\PYG{g+go}{  23.43034408  23.51028152  23.53568488  23.47283308  23.40504525}
\PYG{g+go}{  23.39140066  23.26205508  23.23259052  23.08121043  22.9155358 ]}
\PYG{g+gp}{\PYGZgt{}\PYGZgt{}\PYGZgt{} }\PYG{k}{print}\PYG{p}{(}\PYG{n}{tai}\PYG{o}{.}\PYG{n}{ema}\PYG{p}{(}\PYG{n}{prices}\PYG{p}{,} \PYG{n}{period}\PYG{p}{,} \PYG{l+m+mi}{2}\PYG{p}{)}\PYG{p}{)}
\PYG{g+go}{[ 22.28588695  22.174706    22.35085492  22.37470018  22.5672175}
\PYG{g+go}{  23.21585701  23.89833692  23.77696963  23.82035739  23.9264279}
\PYG{g+go}{  23.68389526  23.79525297  23.85640891  23.68752817  23.28045894}
\PYG{g+go}{  23.13280996  23.29414649  22.79166223  23.04393782  22.51707883}
\PYG{g+go}{  22.23310448]}
\end{Verbatim}

\end{fulllineitems}

\index{ma\_env() (in module technical\_indicators.technical\_indicators)}

\begin{fulllineitems}
\phantomsection\label{reference:technical_indicators.technical_indicators.ma_env}\pysiglinewithargsret{\code{technical\_indicators.technical\_indicators.}\bfcode{ma\_env}}{\emph{prices}, \emph{period}, \emph{percent}, \emph{ma\_type=0}}{}
Moving Average Envelopes are percentage-based envelopes set above and below
a moving average.
They can be used as a trend following indicator.
The envelopes can also be used to identify overbought and oversold levels
when the trend is relatively flat.

Upper Envelope: MA + (MA x percent)
Lower Envelope: MA - (MA x percent)

\href{http://www.csidata.com/?page\_id=797}{http://www.csidata.com/?page\_id=797}

\href{http://stockcharts.com/school/doku.php?id=chart\_school:technical\_indicators:moving\_average\_envel}{http://stockcharts.com/school/doku.php?id=chart\_school:technical\_indicators:moving\_average\_envel}

\href{http://stockcharts.com/school/doku.php?id=chart\_school:technical\_indicators:bollinger\_band\_perce}{http://stockcharts.com/school/doku.php?id=chart\_school:technical\_indicators:bollinger\_band\_perce}
\begin{description}
\item[{Input:}] \leavevmode
prices ndarray
period int \textgreater{} 1 and \textless{} len(prices)
percent float \textgreater{} 0.00 and \textless{} 1.00
ma\_type 0=EMA type 0, 1=EMA type 1, 2=EMA type 2, 3=WMA, 4=SMA

\item[{Output:}] \leavevmode
ma\_envs ndarray with upper, middle, lower bands, range and \%B

\end{description}

Test:

\begin{Verbatim}[commandchars=\\\{\}]
\PYG{g+gp}{\PYGZgt{}\PYGZgt{}\PYGZgt{} }\PYG{k+kn}{import} \PYG{n+nn}{numpy} \PYG{k+kn}{as} \PYG{n+nn}{np}
\PYG{g+gp}{\PYGZgt{}\PYGZgt{}\PYGZgt{} }\PYG{k+kn}{import} \PYG{n+nn}{technical\PYGZus{}indicators} \PYG{k+kn}{as} \PYG{n+nn}{tai}
\PYG{g+gp}{\PYGZgt{}\PYGZgt{}\PYGZgt{} }\PYG{n}{prices} \PYG{o}{=} \PYG{n}{np}\PYG{o}{.}\PYG{n}{array}\PYG{p}{(}\PYG{p}{[}\PYG{l+m+mf}{86.16}\PYG{p}{,} \PYG{l+m+mf}{89.09}\PYG{p}{,} \PYG{l+m+mf}{88.78}\PYG{p}{,} \PYG{l+m+mf}{90.32}\PYG{p}{,} \PYG{l+m+mf}{89.07}\PYG{p}{,} \PYG{l+m+mf}{91.15}\PYG{p}{,} \PYG{l+m+mf}{89.44}\PYG{p}{,}
\PYG{g+gp}{... }\PYG{l+m+mf}{89.18}\PYG{p}{,} \PYG{l+m+mf}{86.93}\PYG{p}{,} \PYG{l+m+mf}{87.68}\PYG{p}{,} \PYG{l+m+mf}{86.96}\PYG{p}{,} \PYG{l+m+mf}{89.43}\PYG{p}{,} \PYG{l+m+mf}{89.32}\PYG{p}{,} \PYG{l+m+mf}{88.72}\PYG{p}{,} \PYG{l+m+mf}{87.45}\PYG{p}{,} \PYG{l+m+mf}{87.26}\PYG{p}{,} \PYG{l+m+mf}{89.50}\PYG{p}{,}
\PYG{g+gp}{... }\PYG{l+m+mf}{87.90}\PYG{p}{,} \PYG{l+m+mf}{89.13}\PYG{p}{,} \PYG{l+m+mf}{90.70}\PYG{p}{,} \PYG{l+m+mf}{92.90}\PYG{p}{,} \PYG{l+m+mf}{92.98}\PYG{p}{,} \PYG{l+m+mf}{91.80}\PYG{p}{,} \PYG{l+m+mf}{92.66}\PYG{p}{,} \PYG{l+m+mf}{92.68}\PYG{p}{,} \PYG{l+m+mf}{92.30}\PYG{p}{,} \PYG{l+m+mf}{92.77}\PYG{p}{,}
\PYG{g+gp}{... }\PYG{l+m+mf}{92.54}\PYG{p}{,} \PYG{l+m+mf}{92.95}\PYG{p}{,} \PYG{l+m+mf}{93.20}\PYG{p}{,} \PYG{l+m+mf}{91.07}\PYG{p}{,} \PYG{l+m+mf}{89.83}\PYG{p}{,} \PYG{l+m+mf}{89.74}\PYG{p}{,} \PYG{l+m+mf}{90.40}\PYG{p}{,} \PYG{l+m+mf}{90.74}\PYG{p}{,} \PYG{l+m+mf}{88.02}\PYG{p}{,} \PYG{l+m+mf}{88.09}\PYG{p}{,}
\PYG{g+gp}{... }\PYG{l+m+mf}{88.84}\PYG{p}{,} \PYG{l+m+mf}{90.78}\PYG{p}{,} \PYG{l+m+mf}{90.54}\PYG{p}{,} \PYG{l+m+mf}{91.39}\PYG{p}{,} \PYG{l+m+mf}{90.65}\PYG{p}{]}\PYG{p}{)}
\PYG{g+gp}{\PYGZgt{}\PYGZgt{}\PYGZgt{} }\PYG{n}{period} \PYG{o}{=} \PYG{l+m+mi}{20}
\PYG{g+gp}{\PYGZgt{}\PYGZgt{}\PYGZgt{} }\PYG{k}{print}\PYG{p}{(}\PYG{n}{tai}\PYG{o}{.}\PYG{n}{ma\PYGZus{}env}\PYG{p}{(}\PYG{n}{prices}\PYG{p}{,} \PYG{n}{period}\PYG{p}{,} \PYG{l+m+mf}{0.1}\PYG{p}{,} \PYG{l+m+mi}{4}\PYG{p}{)}\PYG{p}{)}
\PYG{g+go}{[[  97.57935      88.7085       79.83765      17.7417        0.35635537]}
\PYG{g+go}{ [  97.95005      89.0455       80.14095      17.8091        0.50249872]}
\PYG{g+go}{ [  98.164        89.24         80.316        17.848         0.4742268 ]}
\PYG{g+go}{ [  98.3301       89.391        80.4519       17.8782        0.55196273]}
\PYG{g+go}{ [  98.4588       89.508        80.5572       17.9016        0.47553291]}
\PYG{g+go}{ [  98.65735      89.6885       80.71965      17.9377        0.58147644]}
\PYG{g+go}{ [  98.7206       89.746        80.7714       17.9492        0.48295189]}
\PYG{g+go}{ [  98.90375      89.9125       80.92125      17.9825        0.45926595]}
\PYG{g+go}{ [  99.08855      90.0805       81.07245      18.0161        0.32512863]}
\PYG{g+go}{ [  99.41965      90.3815       81.34335      18.0763        0.35055017]}
\PYG{g+go}{ [  99.72325      90.6575       81.59175      18.1315        0.29607313]}
\PYG{g+go}{ [  99.9493       90.863        81.7767       18.1726        0.42114502]}
\PYG{g+go}{ [  99.9713       90.883        81.7947       18.1766        0.41401032]}
\PYG{g+go}{ [  99.9944       90.904        81.8136       18.1808        0.37987327]}
\PYG{g+go}{ [ 100.0868       90.988        81.8892       18.1976        0.30557876]}
\PYG{g+go}{ [ 100.26775      91.1525       82.03725      18.2305        0.28648419]}
\PYG{g+go}{ [ 100.30955      91.1905       82.07145      18.2381        0.40730942]}
\PYG{g+go}{ [ 100.232        91.12         82.008        18.224         0.32330992]}
\PYG{g+go}{ [ 100.2837       91.167        82.0503       18.2334        0.38828194]}
\PYG{g+go}{ [ 100.37445      91.2495       82.12455      18.2499        0.46989025]}
\PYG{g+go}{ [ 100.36565      91.2415       82.11735      18.2483        0.59088518]}
\PYG{g+go}{ [ 100.2826       91.166        82.0494       18.2332        0.59948884]}
\PYG{g+go}{ [ 100.15445      91.0495       81.94455      18.2099        0.54121385]]}
\end{Verbatim}

\end{fulllineitems}

\index{roc() (in module technical\_indicators.technical\_indicators)}

\begin{fulllineitems}
\phantomsection\label{reference:technical_indicators.technical_indicators.roc}\pysiglinewithargsret{\code{technical\_indicators.technical\_indicators.}\bfcode{roc}}{\emph{prices}, \emph{period=21}}{}
The Rate-of-Change (ROC) indicator, a.k.a. Momentum, is a pure momentum
oscillator that measures the percent change in price from one period to the
next.
The plot forms an oscillator that fluctuates above and below the zero line
as the Rate-of-Change moves from positive to negative.
ROC signals include centerline crossovers, divergences and
overbought-oversold readings. Identifying overbought or oversold extremes
comes natural to the Rate-of-Change oscillator.
It can be used to measure the ROC of any data series, such as price or
another indicator.
Also known as PROC when used with price.

ROC = {[}(Close - Close n periods ago) / (Close n periods ago){]} * 100

\href{http://www.csidata.com/?page\_id=797}{http://www.csidata.com/?page\_id=797}
\href{http://stockcharts.com/school/doku.php?id=chart\_school:technical\_indicators:rate\_of\_change\_roc\_a}{http://stockcharts.com/school/doku.php?id=chart\_school:technical\_indicators:rate\_of\_change\_roc\_a}
\begin{description}
\item[{Input:}] \leavevmode
prices ndarray
period int \textgreater{} 1 and \textless{} len(prices) (optional and defaults to 21)

\item[{Output:}] \leavevmode
rocs ndarray

\end{description}

Test:

\begin{Verbatim}[commandchars=\\\{\}]
\PYG{g+gp}{\PYGZgt{}\PYGZgt{}\PYGZgt{} }\PYG{k+kn}{import} \PYG{n+nn}{numpy} \PYG{k+kn}{as} \PYG{n+nn}{np}
\PYG{g+gp}{\PYGZgt{}\PYGZgt{}\PYGZgt{} }\PYG{k+kn}{import} \PYG{n+nn}{technical\PYGZus{}indicators} \PYG{k+kn}{as} \PYG{n+nn}{tai}
\PYG{g+gp}{\PYGZgt{}\PYGZgt{}\PYGZgt{} }\PYG{n}{prices} \PYG{o}{=} \PYG{n}{np}\PYG{o}{.}\PYG{n}{array}\PYG{p}{(}\PYG{p}{[}\PYG{l+m+mf}{11045.27}\PYG{p}{,} \PYG{l+m+mf}{11167.32}\PYG{p}{,} \PYG{l+m+mf}{11008.61}\PYG{p}{,} \PYG{l+m+mf}{11151.83}\PYG{p}{,} \PYG{l+m+mf}{10926.77}\PYG{p}{,}
\PYG{g+gp}{... }\PYG{l+m+mf}{10868.12}\PYG{p}{,} \PYG{l+m+mf}{10520.32}\PYG{p}{,} \PYG{l+m+mf}{10380.43}\PYG{p}{,} \PYG{l+m+mf}{10785.14}\PYG{p}{,} \PYG{l+m+mf}{10748.26}\PYG{p}{,} \PYG{l+m+mf}{10896.91}\PYG{p}{,} \PYG{l+m+mf}{10782.95}\PYG{p}{,}
\PYG{g+gp}{... }\PYG{l+m+mf}{10620.16}\PYG{p}{,} \PYG{l+m+mf}{10625.83}\PYG{p}{,} \PYG{l+m+mf}{10510.95}\PYG{p}{,} \PYG{l+m+mf}{10444.37}\PYG{p}{,} \PYG{l+m+mf}{10068.01}\PYG{p}{,} \PYG{l+m+mf}{10193.39}\PYG{p}{,} \PYG{l+m+mf}{10066.57}\PYG{p}{,}
\PYG{g+gp}{... }\PYG{l+m+mf}{10043.75}\PYG{p}{]}\PYG{p}{)}
\PYG{g+gp}{\PYGZgt{}\PYGZgt{}\PYGZgt{} }\PYG{k}{print}\PYG{p}{(}\PYG{n}{tai}\PYG{o}{.}\PYG{n}{roc}\PYG{p}{(}\PYG{n}{prices}\PYG{p}{,} \PYG{n}{period}\PYG{o}{=}\PYG{l+m+mi}{12}\PYG{p}{)}\PYG{p}{)}
\PYG{g+go}{[\PYGZhy{}3.84879682 \PYGZhy{}4.84888048 \PYGZhy{}4.52064339 \PYGZhy{}6.34389154 \PYGZhy{}7.85923013 \PYGZhy{}6.20834146}
\PYG{g+go}{ \PYGZhy{}4.31308173 \PYGZhy{}3.24341092]}
\end{Verbatim}

\end{fulllineitems}

\index{rsi() (in module technical\_indicators.technical\_indicators)}

\begin{fulllineitems}
\phantomsection\label{reference:technical_indicators.technical_indicators.rsi}\pysiglinewithargsret{\code{technical\_indicators.technical\_indicators.}\bfcode{rsi}}{\emph{prices}, \emph{period=14}}{}
The Relative Strength Index (RSI) is a momentum oscillator.
It oscillates between 0 and 100.
It is considered overbought/oversold when it's over 70/below 30.
Some traders use 80/20 to be on the safe side.
RSI becomes more accurate as the calculation period (min\_periods)
increases.
This can be lowered to increase sensitivity or raised to decrease
sensitivity.
10-day RSI is more likely to reach overbought or oversold levels than
20-day RSI. The look-back parameters also depend on a security's
volatility.

Like many momentum oscillators, overbought and oversold readings for RSI
work best when prices move sideways within a range.

You can also look for divergence with price.
If the price has new highs/lows, and the RSI hasn't, expect a reversal.
Signals can also be generated by looking for failure swings and centerline
crossovers.

RSI can also be used to identify the general trend.

The RSI was developed by J. Welles Wilder and was first introduced in his
article in the June, 1978 issue of Commodities magazine, now known as
Futures magazine. It is detailed in his book New Concepts In Technical
Trading Systems.

\href{http://www.csidata.com/?page\_id=797}{http://www.csidata.com/?page\_id=797}
\href{http://stockcharts.com/help/doku.php?id=chart\_school:technical\_indicators:relative\_strength\_in}{http://stockcharts.com/help/doku.php?id=chart\_school:technical\_indicators:relative\_strength\_in}
\begin{description}
\item[{Input:}] \leavevmode
prices ndarray
period int \textgreater{} 1 and \textless{} len(prices) (optional and defaults to 14)

\item[{Output:}] \leavevmode
rsis ndarray

\end{description}

Test:

\begin{Verbatim}[commandchars=\\\{\}]
\PYG{g+gp}{\PYGZgt{}\PYGZgt{}\PYGZgt{} }\PYG{k+kn}{import} \PYG{n+nn}{numpy} \PYG{k+kn}{as} \PYG{n+nn}{np}
\PYG{g+gp}{\PYGZgt{}\PYGZgt{}\PYGZgt{} }\PYG{k+kn}{import} \PYG{n+nn}{technical\PYGZus{}indicators} \PYG{k+kn}{as} \PYG{n+nn}{tai}
\PYG{g+gp}{\PYGZgt{}\PYGZgt{}\PYGZgt{} }\PYG{n}{prices} \PYG{o}{=} \PYG{n}{np}\PYG{o}{.}\PYG{n}{array}\PYG{p}{(}\PYG{p}{[}\PYG{l+m+mf}{44.55}\PYG{p}{,} \PYG{l+m+mf}{44.3}\PYG{p}{,} \PYG{l+m+mf}{44.36}\PYG{p}{,} \PYG{l+m+mf}{43.82}\PYG{p}{,} \PYG{l+m+mf}{44.46}\PYG{p}{,} \PYG{l+m+mf}{44.96}\PYG{p}{,} \PYG{l+m+mf}{45.23}\PYG{p}{,}
\PYG{g+gp}{... }\PYG{l+m+mf}{45.56}\PYG{p}{,} \PYG{l+m+mf}{45.98}\PYG{p}{,} \PYG{l+m+mf}{46.22}\PYG{p}{,} \PYG{l+m+mf}{46.03}\PYG{p}{,} \PYG{l+m+mf}{46.17}\PYG{p}{,} \PYG{l+m+mf}{45.75}\PYG{p}{,} \PYG{l+m+mf}{46.42}\PYG{p}{,} \PYG{l+m+mf}{46.42}\PYG{p}{,} \PYG{l+m+mf}{46.14}\PYG{p}{,} \PYG{l+m+mf}{46.17}\PYG{p}{,}
\PYG{g+gp}{... }\PYG{l+m+mf}{46.55}\PYG{p}{,} \PYG{l+m+mf}{46.36}\PYG{p}{,} \PYG{l+m+mf}{45.78}\PYG{p}{,} \PYG{l+m+mf}{46.35}\PYG{p}{,} \PYG{l+m+mf}{46.39}\PYG{p}{,} \PYG{l+m+mf}{45.85}\PYG{p}{,} \PYG{l+m+mf}{46.59}\PYG{p}{,} \PYG{l+m+mf}{45.92}\PYG{p}{,} \PYG{l+m+mf}{45.49}\PYG{p}{,} \PYG{l+m+mf}{44.16}\PYG{p}{,}
\PYG{g+gp}{... }\PYG{l+m+mf}{44.31}\PYG{p}{,} \PYG{l+m+mf}{44.35}\PYG{p}{,} \PYG{l+m+mf}{44.7}\PYG{p}{,} \PYG{l+m+mf}{43.55}\PYG{p}{,} \PYG{l+m+mf}{42.79}\PYG{p}{,} \PYG{l+m+mf}{43.26}\PYG{p}{]}\PYG{p}{)}
\PYG{g+gp}{\PYGZgt{}\PYGZgt{}\PYGZgt{} }\PYG{k}{print}\PYG{p}{(}\PYG{n}{tai}\PYG{o}{.}\PYG{n}{rsi}\PYG{p}{(}\PYG{n}{prices}\PYG{p}{)}\PYG{p}{)}
\PYG{g+go}{[ 70.02141328  65.77440817  66.01226849  68.95536568  65.88342192}
\PYG{g+go}{  57.46707948  62.532685    62.86690858  55.64975092  62.07502976}
\PYG{g+go}{  54.39159393  50.10513101  39.68712141  41.17273382  41.5859395}
\PYG{g+go}{  45.21224077  37.06939108  32.85768734  37.58081218]}
\end{Verbatim}

\end{fulllineitems}

\index{sma() (in module technical\_indicators.technical\_indicators)}

\begin{fulllineitems}
\phantomsection\label{reference:technical_indicators.technical_indicators.sma}\pysiglinewithargsret{\code{technical\_indicators.technical\_indicators.}\bfcode{sma}}{\emph{prices}, \emph{period}}{}
Simple Moving Average (SMA) are used to smooth the data in an array to help
eliminate noise and identify trends.
In SMA, each value in the time period carries equal weight.

They do not predict price direction, but can be used to identify the
direction of the trend or define potential support and resistance levels.

SMA = (P1 + P2 + ... + Pn) / K
where K = n and Pn is the most recent price

\href{http://www.financialwebring.org/gummy-stuff/MA-stuff.htm}{http://www.financialwebring.org/gummy-stuff/MA-stuff.htm}

\href{http://www.csidata.com/?page\_id=797}{http://www.csidata.com/?page\_id=797}
\href{http://stockcharts.com/school/doku.php?st=moving+average\&id=chart\_school:technical\_indicators:moving\_averages}{http://stockcharts.com/school/doku.php?st=moving+average\&id=chart\_school:technical\_indicators:moving\_averages}
\begin{description}
\item[{Input:}] \leavevmode
prices ndarray
period int \textgreater{} 1 and \textless{} len(prices)

\item[{Output:}] \leavevmode
smas ndarray

\end{description}

Test:

\begin{Verbatim}[commandchars=\\\{\}]
\PYG{g+gp}{\PYGZgt{}\PYGZgt{}\PYGZgt{} }\PYG{k+kn}{import} \PYG{n+nn}{numpy} \PYG{k+kn}{as} \PYG{n+nn}{np}
\PYG{g+gp}{\PYGZgt{}\PYGZgt{}\PYGZgt{} }\PYG{k+kn}{import} \PYG{n+nn}{technical\PYGZus{}indicators} \PYG{k+kn}{as} \PYG{n+nn}{tai}
\PYG{g+gp}{\PYGZgt{}\PYGZgt{}\PYGZgt{} }\PYG{n}{prices} \PYG{o}{=} \PYG{n}{np}\PYG{o}{.}\PYG{n}{array}\PYG{p}{(}\PYG{p}{[}\PYG{l+m+mf}{22.27}\PYG{p}{,} \PYG{l+m+mf}{22.19}\PYG{p}{,} \PYG{l+m+mf}{22.08}\PYG{p}{,} \PYG{l+m+mf}{22.17}\PYG{p}{,} \PYG{l+m+mf}{22.18}\PYG{p}{,} \PYG{l+m+mf}{22.13}\PYG{p}{,} \PYG{l+m+mf}{22.23}\PYG{p}{,}
\PYG{g+gp}{... }\PYG{l+m+mf}{22.43}\PYG{p}{,} \PYG{l+m+mf}{22.24}\PYG{p}{,} \PYG{l+m+mf}{22.29}\PYG{p}{,} \PYG{l+m+mf}{22.15}\PYG{p}{,} \PYG{l+m+mf}{22.39}\PYG{p}{,} \PYG{l+m+mf}{22.38}\PYG{p}{,} \PYG{l+m+mf}{22.61}\PYG{p}{,} \PYG{l+m+mf}{23.36}\PYG{p}{,} \PYG{l+m+mf}{24.05}\PYG{p}{,} \PYG{l+m+mf}{23.75}\PYG{p}{,}
\PYG{g+gp}{... }\PYG{l+m+mf}{23.83}\PYG{p}{,} \PYG{l+m+mf}{23.95}\PYG{p}{,} \PYG{l+m+mf}{23.63}\PYG{p}{,} \PYG{l+m+mf}{23.82}\PYG{p}{,} \PYG{l+m+mf}{23.87}\PYG{p}{,} \PYG{l+m+mf}{23.65}\PYG{p}{,} \PYG{l+m+mf}{23.19}\PYG{p}{,} \PYG{l+m+mf}{23.10}\PYG{p}{,} \PYG{l+m+mf}{23.33}\PYG{p}{,} \PYG{l+m+mf}{22.68}\PYG{p}{,}
\PYG{g+gp}{... }\PYG{l+m+mf}{23.10}\PYG{p}{,} \PYG{l+m+mf}{22.40}\PYG{p}{,} \PYG{l+m+mf}{22.17}\PYG{p}{]}\PYG{p}{)}
\PYG{g+gp}{\PYGZgt{}\PYGZgt{}\PYGZgt{} }\PYG{n}{period} \PYG{o}{=} \PYG{l+m+mi}{10}
\PYG{g+gp}{\PYGZgt{}\PYGZgt{}\PYGZgt{} }\PYG{k}{print}\PYG{p}{(}\PYG{n}{tai}\PYG{o}{.}\PYG{n}{sma}\PYG{p}{(}\PYG{n}{prices}\PYG{p}{,} \PYG{n}{period}\PYG{p}{)}\PYG{p}{)}
\PYG{g+go}{[ 22.221  22.209  22.229  22.259  22.303  22.421  22.613  22.765  22.905}
\PYG{g+go}{  23.076  23.21   23.377  23.525  23.652  23.71   23.684  23.612  23.505}
\PYG{g+go}{  23.432  23.277  23.131]}
\end{Verbatim}

\end{fulllineitems}

\index{wma() (in module technical\_indicators.technical\_indicators)}

\begin{fulllineitems}
\phantomsection\label{reference:technical_indicators.technical_indicators.wma}\pysiglinewithargsret{\code{technical\_indicators.technical\_indicators.}\bfcode{wma}}{\emph{prices}, \emph{period}}{}
Weighted Moving Average (WMA) is a type of moving average that assigns a
higher weighting to recent price data.

WMA = (P1 + 2 P2 + 3 P3 + ... + n Pn) / K
where K = (1+2+...+n) = n(n+1)/2 and Pn is the most recent price after the
1st WMA we can use another formula
WMAn = WMAn-1 + w.(Pn - SMA(prices, n-1))
where w = 2 / (n + 1)

\href{http://www.csidata.com/?page\_id=797}{http://www.csidata.com/?page\_id=797}

\href{http://www.financialwebring.org/gummy-stuff/MA-stuff.htm}{http://www.financialwebring.org/gummy-stuff/MA-stuff.htm}

\href{http://www.investopedia.com/terms/l/linearlyweightedmovingaverage.asp}{http://www.investopedia.com/terms/l/linearlyweightedmovingaverage.asp}

\href{http://fxtrade.oanda.com/learn/forex-indicators/weighted-moving-average}{http://fxtrade.oanda.com/learn/forex-indicators/weighted-moving-average}
\begin{description}
\item[{Input:}] \leavevmode
prices ndarray
period int \textgreater{} 1 and \textless{} len(prices)

\item[{Output:}] \leavevmode
wmas ndarray

\end{description}

Test:

\begin{Verbatim}[commandchars=\\\{\}]
\PYG{g+gp}{\PYGZgt{}\PYGZgt{}\PYGZgt{} }\PYG{k+kn}{import} \PYG{n+nn}{numpy} \PYG{k+kn}{as} \PYG{n+nn}{np}
\PYG{g+gp}{\PYGZgt{}\PYGZgt{}\PYGZgt{} }\PYG{k+kn}{import} \PYG{n+nn}{technical\PYGZus{}indicators} \PYG{k+kn}{as} \PYG{n+nn}{tai}
\PYG{g+gp}{\PYGZgt{}\PYGZgt{}\PYGZgt{} }\PYG{n}{prices} \PYG{o}{=} \PYG{n}{np}\PYG{o}{.}\PYG{n}{array}\PYG{p}{(}\PYG{p}{[}\PYG{l+m+mi}{77}\PYG{p}{,} \PYG{l+m+mi}{79}\PYG{p}{,} \PYG{l+m+mi}{79}\PYG{p}{,} \PYG{l+m+mi}{81}\PYG{p}{,} \PYG{l+m+mi}{83}\PYG{p}{,} \PYG{l+m+mi}{49}\PYG{p}{,} \PYG{l+m+mi}{55}\PYG{p}{]}\PYG{p}{)}
\PYG{g+gp}{\PYGZgt{}\PYGZgt{}\PYGZgt{} }\PYG{n}{period} \PYG{o}{=} \PYG{l+m+mi}{5}
\PYG{g+gp}{\PYGZgt{}\PYGZgt{}\PYGZgt{} }\PYG{k}{print}\PYG{p}{(}\PYG{n}{tai}\PYG{o}{.}\PYG{n}{wma}\PYG{p}{(}\PYG{n}{prices}\PYG{p}{,} \PYG{n}{period}\PYG{p}{)}\PYG{p}{)}
\PYG{g+go}{[ 80.73333333  70.46666667  64.06666667]}
\end{Verbatim}

\end{fulllineitems}



\chapter{License}
\label{license::doc}\label{license:license}
\begin{Verbatim}[commandchars=\\\{\}]
    technical\PYGZus{}indicators \PYGZhy{} blablabla
    Copyright (C) 2014  Joao Carlos Roseta Matos

    This program is free software; you can redistribute it and/or modify
    it under the terms of the GNU General Public License as published by
    the Free Software Foundation; either version 2 of the License, or
    (at your option) any later version.

    This program is distributed in the hope that it will be useful,
    but WITHOUT ANY WARRANTY; without even the implied warranty of
    MERCHANTABILITY or FITNESS FOR A PARTICULAR PURPOSE.  See the
    GNU General Public License for more details.

    You should have received a copy of the GNU General Public License along
    with this program; if not, write to the Free Software Foundation, Inc.,
    51 Franklin Street, Fifth Floor, Boston, MA 02110\PYGZhy{}1301 USA.



                    GNU GENERAL PUBLIC LICENSE
                       Version 2, June 1991

 Copyright (C) 1989, 1991 Free Software Foundation, Inc., \PYGZlt{}http://fsf.org/\PYGZgt{}
 51 Franklin Street, Fifth Floor, Boston, MA 02110\PYGZhy{}1301 USA
 Everyone is permitted to copy and distribute verbatim copies
 of this license document, but changing it is not allowed.

                            Preamble

  The licenses for most software are designed to take away your
freedom to share and change it.  By contrast, the GNU General Public
License is intended to guarantee your freedom to share and change free
software\PYGZhy{}\PYGZhy{}to make sure the software is free for all its users.  This
General Public License applies to most of the Free Software
Foundation\PYGZsq{}s software and to any other program whose authors commit to
using it.  (Some other Free Software Foundation software is covered by
the GNU Lesser General Public License instead.)  You can apply it to
your programs, too.

  When we speak of free software, we are referring to freedom, not
price.  Our General Public Licenses are designed to make sure that you
have the freedom to distribute copies of free software (and charge for
this service if you wish), that you receive source code or can get it
if you want it, that you can change the software or use pieces of it
in new free programs; and that you know you can do these things.

  To protect your rights, we need to make restrictions that forbid
anyone to deny you these rights or to ask you to surrender the rights.
These restrictions translate to certain responsibilities for you if you
distribute copies of the software, or if you modify it.

  For example, if you distribute copies of such a program, whether
gratis or for a fee, you must give the recipients all the rights that
you have.  You must make sure that they, too, receive or can get the
source code.  And you must show them these terms so they know their
rights.

  We protect your rights with two steps: (1) copyright the software, and
(2) offer you this license which gives you legal permission to copy,
distribute and/or modify the software.

  Also, for each author\PYGZsq{}s protection and ours, we want to make certain
that everyone understands that there is no warranty for this free
software.  If the software is modified by someone else and passed on, we
want its recipients to know that what they have is not the original, so
that any problems introduced by others will not reflect on the original
authors\PYGZsq{} reputations.

  Finally, any free program is threatened constantly by software
patents.  We wish to avoid the danger that redistributors of a free
program will individually obtain patent licenses, in effect making the
program proprietary.  To prevent this, we have made it clear that any
patent must be licensed for everyone\PYGZsq{}s free use or not licensed at all.

  The precise terms and conditions for copying, distribution and
modification follow.

                    GNU GENERAL PUBLIC LICENSE
   TERMS AND CONDITIONS FOR COPYING, DISTRIBUTION AND MODIFICATION

  0. This License applies to any program or other work which contains
a notice placed by the copyright holder saying it may be distributed
under the terms of this General Public License.  The \PYGZdq{}Program\PYGZdq{}, below,
refers to any such program or work, and a \PYGZdq{}work based on the Program\PYGZdq{}
means either the Program or any derivative work under copyright law:
that is to say, a work containing the Program or a portion of it,
either verbatim or with modifications and/or translated into another
language.  (Hereinafter, translation is included without limitation in
the term \PYGZdq{}modification\PYGZdq{}.)  Each licensee is addressed as \PYGZdq{}you\PYGZdq{}.

Activities other than copying, distribution and modification are not
covered by this License; they are outside its scope.  The act of
running the Program is not restricted, and the output from the Program
is covered only if its contents constitute a work based on the
Program (independent of having been made by running the Program).
Whether that is true depends on what the Program does.

  1. You may copy and distribute verbatim copies of the Program\PYGZsq{}s
source code as you receive it, in any medium, provided that you
conspicuously and appropriately publish on each copy an appropriate
copyright notice and disclaimer of warranty; keep intact all the
notices that refer to this License and to the absence of any warranty;
and give any other recipients of the Program a copy of this License
along with the Program.

You may charge a fee for the physical act of transferring a copy, and
you may at your option offer warranty protection in exchange for a fee.

  2. You may modify your copy or copies of the Program or any portion
of it, thus forming a work based on the Program, and copy and
distribute such modifications or work under the terms of Section 1
above, provided that you also meet all of these conditions:

    a) You must cause the modified files to carry prominent notices
    stating that you changed the files and the date of any change.

    b) You must cause any work that you distribute or publish, that in
    whole or in part contains or is derived from the Program or any
    part thereof, to be licensed as a whole at no charge to all third
    parties under the terms of this License.

    c) If the modified program normally reads commands interactively
    when run, you must cause it, when started running for such
    interactive use in the most ordinary way, to print or display an
    announcement including an appropriate copyright notice and a
    notice that there is no warranty (or else, saying that you provide
    a warranty) and that users may redistribute the program under
    these conditions, and telling the user how to view a copy of this
    License.  (Exception: if the Program itself is interactive but
    does not normally print such an announcement, your work based on
    the Program is not required to print an announcement.)

These requirements apply to the modified work as a whole.  If
identifiable sections of that work are not derived from the Program,
and can be reasonably considered independent and separate works in
themselves, then this License, and its terms, do not apply to those
sections when you distribute them as separate works.  But when you
distribute the same sections as part of a whole which is a work based
on the Program, the distribution of the whole must be on the terms of
this License, whose permissions for other licensees extend to the
entire whole, and thus to each and every part regardless of who wrote it.

Thus, it is not the intent of this section to claim rights or contest
your rights to work written entirely by you; rather, the intent is to
exercise the right to control the distribution of derivative or
collective works based on the Program.

In addition, mere aggregation of another work not based on the Program
with the Program (or with a work based on the Program) on a volume of
a storage or distribution medium does not bring the other work under
the scope of this License.

  3. You may copy and distribute the Program (or a work based on it,
under Section 2) in object code or executable form under the terms of
Sections 1 and 2 above provided that you also do one of the following:

    a) Accompany it with the complete corresponding machine\PYGZhy{}readable
    source code, which must be distributed under the terms of Sections
    1 and 2 above on a medium customarily used for software interchange; or,

    b) Accompany it with a written offer, valid for at least three
    years, to give any third party, for a charge no more than your
    cost of physically performing source distribution, a complete
    machine\PYGZhy{}readable copy of the corresponding source code, to be
    distributed under the terms of Sections 1 and 2 above on a medium
    customarily used for software interchange; or,

    c) Accompany it with the information you received as to the offer
    to distribute corresponding source code.  (This alternative is
    allowed only for noncommercial distribution and only if you
    received the program in object code or executable form with such
    an offer, in accord with Subsection b above.)

The source code for a work means the preferred form of the work for
making modifications to it.  For an executable work, complete source
code means all the source code for all modules it contains, plus any
associated interface definition files, plus the scripts used to
control compilation and installation of the executable.  However, as a
special exception, the source code distributed need not include
anything that is normally distributed (in either source or binary
form) with the major components (compiler, kernel, and so on) of the
operating system on which the executable runs, unless that component
itself accompanies the executable.

If distribution of executable or object code is made by offering
access to copy from a designated place, then offering equivalent
access to copy the source code from the same place counts as
distribution of the source code, even though third parties are not
compelled to copy the source along with the object code.

  4. You may not copy, modify, sublicense, or distribute the Program
except as expressly provided under this License.  Any attempt
otherwise to copy, modify, sublicense or distribute the Program is
void, and will automatically terminate your rights under this License.
However, parties who have received copies, or rights, from you under
this License will not have their licenses terminated so long as such
parties remain in full compliance.

  5. You are not required to accept this License, since you have not
signed it.  However, nothing else grants you permission to modify or
distribute the Program or its derivative works.  These actions are
prohibited by law if you do not accept this License.  Therefore, by
modifying or distributing the Program (or any work based on the
Program), you indicate your acceptance of this License to do so, and
all its terms and conditions for copying, distributing or modifying
the Program or works based on it.

  6. Each time you redistribute the Program (or any work based on the
Program), the recipient automatically receives a license from the
original licensor to copy, distribute or modify the Program subject to
these terms and conditions.  You may not impose any further
restrictions on the recipients\PYGZsq{} exercise of the rights granted herein.
You are not responsible for enforcing compliance by third parties to
this License.

  7. If, as a consequence of a court judgment or allegation of patent
infringement or for any other reason (not limited to patent issues),
conditions are imposed on you (whether by court order, agreement or
otherwise) that contradict the conditions of this License, they do not
excuse you from the conditions of this License.  If you cannot
distribute so as to satisfy simultaneously your obligations under this
License and any other pertinent obligations, then as a consequence you
may not distribute the Program at all.  For example, if a patent
license would not permit royalty\PYGZhy{}free redistribution of the Program by
all those who receive copies directly or indirectly through you, then
the only way you could satisfy both it and this License would be to
refrain entirely from distribution of the Program.

If any portion of this section is held invalid or unenforceable under
any particular circumstance, the balance of the section is intended to
apply and the section as a whole is intended to apply in other
circumstances.

It is not the purpose of this section to induce you to infringe any
patents or other property right claims or to contest validity of any
such claims; this section has the sole purpose of protecting the
integrity of the free software distribution system, which is
implemented by public license practices.  Many people have made
generous contributions to the wide range of software distributed
through that system in reliance on consistent application of that
system; it is up to the author/donor to decide if he or she is willing
to distribute software through any other system and a licensee cannot
impose that choice.

This section is intended to make thoroughly clear what is believed to
be a consequence of the rest of this License.

  8. If the distribution and/or use of the Program is restricted in
certain countries either by patents or by copyrighted interfaces, the
original copyright holder who places the Program under this License
may add an explicit geographical distribution limitation excluding
those countries, so that distribution is permitted only in or among
countries not thus excluded.  In such case, this License incorporates
the limitation as if written in the body of this License.

  9. The Free Software Foundation may publish revised and/or new versions
of the General Public License from time to time.  Such new versions will
be similar in spirit to the present version, but may differ in detail to
address new problems or concerns.

Each version is given a distinguishing version number.  If the Program
specifies a version number of this License which applies to it and \PYGZdq{}any
later version\PYGZdq{}, you have the option of following the terms and conditions
either of that version or of any later version published by the Free
Software Foundation.  If the Program does not specify a version number of
this License, you may choose any version ever published by the Free Software
Foundation.

  10. If you wish to incorporate parts of the Program into other free
programs whose distribution conditions are different, write to the author
to ask for permission.  For software which is copyrighted by the Free
Software Foundation, write to the Free Software Foundation; we sometimes
make exceptions for this.  Our decision will be guided by the two goals
of preserving the free status of all derivatives of our free software and
of promoting the sharing and reuse of software generally.

                            NO WARRANTY

  11. BECAUSE THE PROGRAM IS LICENSED FREE OF CHARGE, THERE IS NO WARRANTY
FOR THE PROGRAM, TO THE EXTENT PERMITTED BY APPLICABLE LAW.  EXCEPT WHEN
OTHERWISE STATED IN WRITING THE COPYRIGHT HOLDERS AND/OR OTHER PARTIES
PROVIDE THE PROGRAM \PYGZdq{}AS IS\PYGZdq{} WITHOUT WARRANTY OF ANY KIND, EITHER EXPRESSED
OR IMPLIED, INCLUDING, BUT NOT LIMITED TO, THE IMPLIED WARRANTIES OF
MERCHANTABILITY AND FITNESS FOR A PARTICULAR PURPOSE.  THE ENTIRE RISK AS
TO THE QUALITY AND PERFORMANCE OF THE PROGRAM IS WITH YOU.  SHOULD THE
PROGRAM PROVE DEFECTIVE, YOU ASSUME THE COST OF ALL NECESSARY SERVICING,
REPAIR OR CORRECTION.

  12. IN NO EVENT UNLESS REQUIRED BY APPLICABLE LAW OR AGREED TO IN WRITING
WILL ANY COPYRIGHT HOLDER, OR ANY OTHER PARTY WHO MAY MODIFY AND/OR
REDISTRIBUTE THE PROGRAM AS PERMITTED ABOVE, BE LIABLE TO YOU FOR DAMAGES,
INCLUDING ANY GENERAL, SPECIAL, INCIDENTAL OR CONSEQUENTIAL DAMAGES ARISING
OUT OF THE USE OR INABILITY TO USE THE PROGRAM (INCLUDING BUT NOT LIMITED
TO LOSS OF DATA OR DATA BEING RENDERED INACCURATE OR LOSSES SUSTAINED BY
YOU OR THIRD PARTIES OR A FAILURE OF THE PROGRAM TO OPERATE WITH ANY OTHER
PROGRAMS), EVEN IF SUCH HOLDER OR OTHER PARTY HAS BEEN ADVISED OF THE
POSSIBILITY OF SUCH DAMAGES.

                     END OF TERMS AND CONDITIONS
\end{Verbatim}


\chapter{ChangeLog}
\label{changelog::doc}\label{changelog:changelog}
0.0.7 2014-05-30

\begin{Verbatim}[commandchars=\\\{\}]
Corrected test and yml files
\end{Verbatim}

0.0.6 2014-05-29

\begin{Verbatim}[commandchars=\\\{\}]
Added Shippable CI
\end{Verbatim}

0.0.5 2014-05-29

\begin{Verbatim}[commandchars=\\\{\}]
Added doctests, packaging, build automation, sphinx doc, travis.
Changed license and versioning.
\end{Verbatim}

0.0.4 2013-07-03

\begin{Verbatim}[commandchars=\\\{\}]
Added ROC and MA envelopes
\end{Verbatim}

0.0.3 2013-06-30

\begin{Verbatim}[commandchars=\\\{\}]
Added WMA and more EMA types.
\end{Verbatim}

0.0.2 2013-06-18

\begin{Verbatim}[commandchars=\\\{\}]
Added Bollinger bandwidth and \PYGZpc{}B
Created a GitHub repository
\end{Verbatim}

0.0.1 2013-06-05

\begin{Verbatim}[commandchars=\\\{\}]
Includes RSI, SMA, EMA and BB
\end{Verbatim}


\chapter{Code authors}
\label{codeauthors:code-authors}\label{codeauthors::doc}
\begin{Verbatim}[commandchars=\\\{\}]
Joao Matos \PYGZlt{}jcrmatos@gmail.com\PYGZgt{}
\end{Verbatim}


\chapter{Indices and tables}
\label{index:indices-and-tables}\begin{itemize}
\item {} 
\emph{genindex}

\item {} 
\emph{modindex}

\item {} 
\emph{search}

\end{itemize}


\renewcommand{\indexname}{Python Module Index}
\begin{theindex}
\def\bigletter#1{{\Large\sffamily#1}\nopagebreak\vspace{1mm}}
\bigletter{t}
\item {\texttt{technical\_indicators.technical\_indicators}}, \pageref{reference:module-technical_indicators.technical_indicators}
\end{theindex}

\renewcommand{\indexname}{Index}
\printindex
\end{document}
